\NeedsTeXFormat{LaTeX2e}[2005/12/01]
%%    2010/04/06 v1.0 Vorlage Master-Forschungspraktikum Versuchsauswertung
%%    based on the 2009/10/14 v0.1 GAUBM template by Prof Pruschke

\documentclass[twoside,        %% zweiseitiges Layout
               BCOR12mm,       %% Bindekorrektur 12 mm
% please comment out if report is in English
               english,ngerman, %% Dokumentspr. Deutsch, Alternativspr. Englisch
% please remove comment if report is in English 
%               ngerman,english, %% Dokumentspr. Englisch, Alternativspr. Deutsch
               fleqn,headsepline=false,footsepline=false
              ]{Vorlage/MFPREPORT}

%% Pakete und Definitionen ausgelagert
\usepackage{a4}
\usepackage{multicol}

% language option set in JGNSUM class
\usepackage{babel}
\usepackage{hyperref}

%% FONT:
%\usepackage{lmodern}
\usepackage{times} % sieht besser aus als lmodern
%\usepackage{palatino} % sieht schlechter aus als times
%\usepackage{mathpazo} % very ugly font, to be loaded later ???
%\usepackage{cmbright} % doesn't work either
\usepackage[T1]{fontenc}
\usepackage{textcomp}

\usepackage{ucs}
\usepackage[utf8x]{inputenc}

\usepackage{amsfonts}
\usepackage{amstext}
\usepackage{amsmath}
\usepackage{amsthm}
\usepackage{amssymb}
\usepackage{amsbsy}   % AMS-Boldsymbol

% \usepackage{mathabx} % e.g. for \Sun
%% but not a standard package (neither texlive nor Miktex)
%% so use wasysym (\astrosun) instead
\usepackage{wasysym} % e.g. for \astrosun or \CheckedBox

\usepackage{bbm,mathrsfs}

\usepackage{textcomp} % noch einige coole symbole

\usepackage{sectsty}
\allsectionsfont{\raggedright}

\usepackage[numbers]{natbib}
\citestyle{dinat}
\bibliographystyle{dinat}

\usepackage{makeidx}

\usepackage{url}	% für hübsche URLs mit Link
\usepackage{color}	% für farben a la \definecolor{Gray}{gray}{0.5}
\usepackage{verbatim}
\usepackage{subfigure}
\usepackage{listings}

\usepackage{fancybox}
%usage:
%\begin{Verbatim}[frame=single,label=Titel]
%Verbatim Zeile
%\end{Verbatim}


 \setlength{\textwidth}{16.2cm}
 \setlength{\textheight}{24cm}
 \setlength{\oddsidemargin}{0cm}
 \setlength{\evensidemargin}{-0.5cm}

 %unbedingt nach abmessungen einfügen!
 \usepackage{fancyhdr}
 \pagestyle{fancy}
 %\sloppy % für weniger absatzfehler

 \setcounter{tocdepth}{2}
 \setcounter{secnumdepth}{2}

 \ifreportelse{\numberwithin{equation}{chapter}}{\numberwithin{equation}{section}}
 \theoremstyle{plain}% default
 \ifreportelse{\newtheorem{thm}{Theorem}[chapter]}{\newtheorem{thm}{Theorem}[section]}

 \newtheorem{satz}{Satz}
 \newtheorem{lem}[thm]{Lemma}
 \newtheorem{prop}[thm]{Proposition}
 \newtheorem{kor}[thm]{Korollar}
 \newtheorem{cor}[thm]{Corollary}

 \theoremstyle{definition}
 \newtheorem{defi}{Definition}

 \def\@proof{%
  \if@englishpreamble{Proof}\else{Beweis}\fi
 }
 \newenvironment{bew}{\begin{proof}[\@proof]}{\end{proof}}



%% einbinden einiger nützlicher Befehle
\newcommand{\iflanggerman}[2]{
 \iflanguage{german}{#1}{
  \iflanguage{ngerman}{#1}{#2}
 }
}

% box around the whole equation, number inclusive
\newcommand{\boxedeqn}[1]{%
  \[\fbox{%
      \addtolength{\linewidth}{-2\fboxsep}%
      \addtolength{\linewidth}{-2\fboxrule}%
      \begin{minipage}{\linewidth}%
      \begin{equation}#1\end{equation}%
      \end{minipage}%
    }\]%
}

\iflanggerman{
 \newcommand{\const}{\mathrm{konst}}
 \newcommand{\Const}{\mathrm{konst.}}
}{
 \newcommand{\const}{\mathrm{const}}
 \newcommand{\Const}{\mathrm{const.}}
}

% von Meier
\newcommand{\nbd}{\nobreakdash-\hspace{0pt}}
% example: $K$\nbd{}Vektorraum
\newcommand*{\transpose}[1]{\prescript{t}{}{#1}}
\newcommand*{\conjugate}[1]{\overline{#1}}
\newcommand*{\abs}[1]{\lvert#1\rvert}
\newcommand*{\Mod}{\mathrm{mod}}
\newcommand{\symdif}{\mathbin\triangle}
\DeclareMathOperator{\Graph}{Graph}
\DeclareMathOperator{\id}{id}
\DeclareMathOperator*{\grad}{grad}
\DeclareMathOperator*{\Div}{div}
\DeclareMathOperator*{\rot}{rot}
\DeclareMathOperator{\sig}{sig}
\DeclareMathOperator{\sgn}{sgn}
\DeclareMathOperator{\diag}{diag}
\DeclareMathOperator{\tr}{tr}
\DeclareMathOperator{\Sp}{Sp}
\DeclareMathOperator{\im}{Im}
\DeclareMathOperator{\re}{Re}

\newcommand{\vcentcolon}{\mathop{:}}



%Zur Formatierung in der Matheumgebung
\renewcommand{\t}{\ensuremath{\rm\tiny}} % Tiefgestellter Text in der Matheumgebung wird schoener mit: $\Phi_{\t{Text}}$
\renewcommand{\d}{\ensuremath{\mathrm{d}}} % Die totale Ableitung ist stets aufrecht zu setzen: \d
\newcommand{\diff}[3][]{\ensuremath{\frac{\d^{#1}#2}{\d#3^{#1}}}} % einfache Ableitung nach x: $\ddx{\Phi}$
\newcommand{\pdiff}[3][]{\ensuremath{\frac{\partial^{#1}#2}{\partial#3^{#1}}}} % wie gesprochen, eine partielle Ableitung: \del
\newcommand{\aeqiv}{\ensuremath{\qquad \Longleftrightarrow \qquad}} % Eine Aequivalenz
\newcommand{\folgt}{\ensuremath{\qquad \Longrightarrow \qquad}} % Ein Folgepfeil mit Abstaenden
\newcommand{\corresponds}{\ensuremath{\mathrel{\widehat{=}}}} % Befehl für "Entspricht"-Zeichen
\newcommand{\mi}[1]{\ensuremath{\mathit{#1}}} % italics für griechische Buchstaben in Matheumgebung

%Um nicht so viel schreiben zu müssen...
\newcommand{\bs}[1]{\boldsymbol{#1}}
\newcommand{\ol}[1]{\overline{#1}}
\newcommand{\wtilde}[1]{\widetilde{#1}}
\newcommand{\mrm}[1]{\mathrm{#1}}
\newcommand{\mbf}[1]{\mathbf{#1}}
\newcommand{\mbb}[1]{\mathbb{#1}}
\newcommand{\mcal}[1]{\mathcal{#1}}
\newcommand{\mfrak}[1]{\mathfrak{#1}}

%Abkürzungen
\newcommand{\zB}{z.\,B.\ }
\newcommand{\bzw}{b.\,z.\, w.\ }
\newcommand{\Dh}{d.\,h.\ }
\newcommand{\Gl}{Gl.\ }
\newcommand{\Abb}{Abb.\ }
\newcommand{\Tab}{Tab.\ }


\begin{document}
\LabratoryName{FM.TES}{Tunneleffekt bei Supraleitern}
\ProtocolAuthor{Eric}{Bertok}{eric.bertok@stud.uni-goettingen.de}
\Assistant{Christoph}{Meyer}
\ResearchFocus{Festkörper- und Materialphysik (M.phy.403)}
% In der naechsten Version beruecksichtigt
%\Collaborator{Vorname}{Nachname}{email}
\ConductedOn{25}{2}{2017}
\date{\today}
% eines von beiden
\CopyNotWanted
%\CopyWanted

\pagenumbering{roman}
\maketitle

%\begin{otherlanguage}{english}
%\end{otherlanguage}

\tableofcontents

\clearpage
\pagenumbering{arabic}

\section{Einleitung}
\label{sec:einleitung}
\section{Theorie}
\label{sec:theorie}
\subsection{Grundlagen: Zustandsdichte, Fermiverteilung, Besetzungsdichte}
Die Zustandsdichte $D(E)$ gibt die Anzahl der erlaubten Mikrozustände pro
Energieintervall $\d E$ an. Sie ist für kontinuierliche Energien allgemein
definiert durch $D(E):=\int \frac{\d^d k}{(2
\pi)^d}\delta\left(E-E(\vec{k})\right)$ [cit],
wobei $\vec k$ die Wellenzahl und $d$ die Dimension (für unseren Fall 3) ist.
$E(\vec k)$ ist die Dispersionsrelation der Teilchen.
Für freie Elektronen ergibt sich [cit]:
\begin{equation}
    \label{eq:freielek}
    D(E)=\frac{\left(2m^*\right)^\frac{3}{2}}{2\pi ^2\hbar ^3}\sqrt{E}, 
\end{equation}
wobei $m^*$ die sog. effektive Masse der Elektronen ist. Diese ist für freie
Elektronen gleich ihrer tatsächlichen Masse, berücksichtigt bei einem
Festkörper aber zusätzlich den Einfluss des Kristallpotentials.

Die Fermi-Dirac Verteilung gibt die mittlere thermische Energiebesetzung von Spin-1/2
Teilchen in Abhängigkeit der Temperatur $T$ und des chemischen Potentials $\mu$
an. Sie lautet $f_T(E)=\frac{1}{\exp{\beta(E-\mu)-1}}$, wobei
$\beta=\frac{1}{k_b T}$ der Boltzmann- Faktor ist. Für $T=0$ ist sie eine
Stufenfunktion [pic]: Unterhalb der sog. Fermienergie $E_F=\mu$ sind alle
Zustände mit Wahrscheinlichkeit 1 besetzt, während alle oberhalb von $E_F$
unbesetzt sind. Dies bezeichnet man als ``Fermikante''.

Die Besetzungsdichte, also die Dichte der tatsächlich besetzten Zustände pro
Energieintervall erhält man als
Produkt aus Zustandsdichte und Fermiverteilung: $n(E)=D(E)f_T(E)$. Hierdurch
erreicht man eine Trennung der systemabhängigen Größen in $D(E)$ und der
systemunabhängigen Thermodynamik in $f_T(E)$.

\subsection{Tunneleffekt}
Der Tunneleffekt ist ein quantenmechanischer Effekt, bei dem ein Teilchen eine
Potentialbarriere endlicher Höhe und breite durchqueren kann, obwohl sie höher
als die Energie des Teilchens ist. Dies kommt dadurch zustande, dass die
Wellenfunktion jenseits der Barriere verschieden von Null ist. 
Durch Lösen der Schrödingergleichung mit einer Kastenbarriere erhält man,
dass die Tunnelwahrscheinlichkeit exponentiell mit der Breite $d$ der Barriere
abfällt: [cit]
\begin{align}
    \label{eq:tunnelamp}
    A_\text{tunnel}\propto e^{-\sqrt{(2m/\hbar^2 (V_0-E)}}e^{-d}.
\end{align}
Jenseits der Barriere das Potential wieder Null. Somit ist die Wellenlänge und damit die Energie des getunnelten
Teilchens unverändert. Lediglich die Wahrscheinlichkeit des Tunnelns nimmt ab.
In unserem Versuch, in dem ein Aluminium- Aluminiumoxid- Blei- Kontakt
verwendet wird, ist die Potentialbreite durch die Dicke der
Aluminiumoxidschicht gegeben. Die Höhe des Potentialwalls ist gegeben durch die
Differenz der Fermienergien der beiden Leiter Al und Pb und der des
Aluminiumoxids.
Weiterhin ist zu beachten, dass ein Teilchen nur dann durch die Barriere
tunneln kann, wenn es auf der anderen Seite freie Zustände vorfindet. Ist dies
nicht der Fall, kommt es zur exponentiellen Abnahme der Amplitude. Auf diese
Weise kann man Schlüsse über die Zustandsdichte der betrachteten Materialien
ziehen, was in diesem Versuch benutzt wird, um die Energielücke von
Supraleitern zu vermessen.

\subsection{mikroskopische Theorie der Supraleitung: Cooper- Paare}
Die BCS Theorie, welche 1957 von Badeen, Cooper und Schrieffer entwickelt
wurde, war die erste erfolgreiche mikroskopische Theorie der Supraleitung
\cite[Kap.\;2]{buckel2013supraleitung}. Die positiven Atomrümpfe werden durch die
negativ geladenen Elektronen polarisiert. Da die Atomrümpfe vergleichsweise
schwer sind, haben sie eine geringe Schwingungsfrequenz. Dadurch kommt es zu
einer retardierten Polarisierung. Ein anderes Elektron spürt diese
Polarisierung also, wenn das erste bereits eine große Distanz zurückgelegt hat.
Aufgrund dessen kann man die Coulombabstoßung zwischen diesen beiden Elektronen
vernachlässigen. Mathematisch lässt sich diese anziehende Wechselwirkung
zwischen zwei Elektronen als ein Austausch eines virtuellen Phonons
beschreiben. Für eine detaillierte Ausarbeitung im Formalismus der zweiten
Quantisierung siehe \zB \cite[Kap.\;10.4]{enss2011tieftemperaturphysik}.
Wenn $\vec q$ der Impuls des Phonons und $\vec k_1$, $\vec k_2$ die der beiden Elektronen
sind, so gilt nach dem Austausch des Phonons
\cite[S.\;308]{enss2011tieftemperaturphysik}
\begin{align}
    \label{eq:phononimp}
    \vec k_1'=\vec k_1 +\vec q, \quad \vec k_2'=\vec k_2 -\vec q,
\end{align}
der Gesamtimpuls $\vec K=\vec k_1 + \vec k_2$ bleibt also erhalten.
Aufgrund der Fermiverteilung sind für die Elektronen nur Energien oberhalb der
Fermikante zugänglich. Approximiert man die Phononenfrequenz mit der
Debyefrequenz $\omega_D$ [cit], so liegen die freien Zustände innerhalb der
Energieschale $[E_F,E_F+\hbar \omega_D]$ oder äquivalent im k-Raum innerhalb
der Schale mit Dicke $\frac{m\omega_D}{\hbar k_F}$, wobei $k_F$ der
Fermiwellenvektor ist. Für beide Elektronen sind diese Schalen in Abb.
\ref{fig:schale} zu sehen. Impulserhaltung ist erfüllt für alle Vektoren im
Überlapp beider Schalen. Dieser ist maximal für $\vec k_1 = -\vec k_2$. Das
Fermi Prinzip verlangt außerdem, dass beide Elektronen entgegengesetzten Spin
besitzen. Das Paar ${\vec k,\uparrow},{-\vec k,\downarrow}$ bezeichnet mnan als
Cooper-Paar (CP). Cooper-Paare sind effektive Bosonen. Durch die Attraktion von
Elektronenpaaren wird die Gesamtenergie verringert. Es kommt zu einer
makroskopischen Besetzung, bei der im Grundzustand alle Cooper-Paare im exakt gleichen
Quantenzustand sind \cite[Kap. 2.2]{buckel2013supraleitung}. Die Gesamtheit
aller Cooper-Paare wird deswegen als eine makroskopische Wellenfunktion
beschrieben.
\subsection{Zustandsdichte und Kennlinien von Supraleitern}
\subsubsection{NL/NL Tunnelkennlinie}
\subsubsection{NL/SL Tunnelkennlinie}
\section{Durchführung}
\label{sec:durchfuehrung}
\section{Auswertung}
\label{sec:auswertung}
\section{Diskussion}

\clearpage

%\appendix
%
%\section{Anhang}
%
%
%\ldots
%
%\clearpage

% nicht unbedingt erforderlich
%\listoffigures
% nicht unbedingt erforderlich
%\listoftables

\bibliography{literatur}

\end{document}

