\NeedsTeXFormat{LaTeX2e}[2005/12/01]
%%    2010/04/06 v1.0 Vorlage Master-Forschungspraktikum Versuchsauswertung
%%    based on the 2009/10/14 v0.1 GAUBM template by Prof Pruschke

\documentclass[twoside,        %% zweiseitiges Layout
               BCOR12mm,       %% Bindekorrektur 12 mm
% please comment out if report is in English
               english,ngerman, %% Dokumentspr. Deutsch, Alternativspr. Englisch
% please remove comment if report is in English 
%               ngerman,english, %% Dokumentspr. Englisch, Alternativspr. Deutsch
               fleqn,headsepline=false,footsepline=false
              ]{Vorlage/MFPREPORT}

%% Pakete und Definitionen ausgelagert
\input{Vorlage/packages}
\input{Vorlage/layout}

%% einbinden einiger nützlicher Befehle
\input{Vorlage/shortcuts}

%Zur Formatierung in der Matheumgebung
\renewcommand{\t}{\ensuremath{\rm\tiny}} % Tiefgestellter Text in der Matheumgebung wird schoener mit: $\Phi_{\t{Text}}$
\renewcommand{\d}{\ensuremath{\mathrm{d}}} % Die totale Ableitung ist stets aufrecht zu setzen: \d
\newcommand{\diff}[3][]{\ensuremath{\frac{\d^{#1}#2}{\d#3^{#1}}}} % einfache Ableitung nach x: $\ddx{\Phi}$
\newcommand{\pdiff}[3][]{\ensuremath{\frac{\partial^{#1}#2}{\partial#3^{#1}}}} % wie gesprochen, eine partielle Ableitung: \del
\newcommand{\aeqiv}{\ensuremath{\qquad \Longleftrightarrow \qquad}} % Eine Aequivalenz
\newcommand{\folgt}{\ensuremath{\qquad \Longrightarrow \qquad}} % Ein Folgepfeil mit Abstaenden
\newcommand{\corresponds}{\ensuremath{\mathrel{\widehat{=}}}} % Befehl für "Entspricht"-Zeichen
\newcommand{\mi}[1]{\ensuremath{\mathit{#1}}} % italics für griechische Buchstaben in Matheumgebung

%Um nicht so viel schreiben zu müssen...
\newcommand{\bs}[1]{\boldsymbol{#1}}
\newcommand{\ol}[1]{\overline{#1}}
\newcommand{\wtilde}[1]{\widetilde{#1}}
\newcommand{\mrm}[1]{\mathrm{#1}}
\newcommand{\mbf}[1]{\mathbf{#1}}
\newcommand{\mbb}[1]{\mathbb{#1}}
\newcommand{\mcal}[1]{\mathcal{#1}}
\newcommand{\mfrak}[1]{\mathfrak{#1}}

%Abkürzungen
\newcommand{\zB}{z.\,B.\ }
\newcommand{\bzw}{b.\,z.\, w.\ }
\newcommand{\Dh}{d.\,h.\ }
\newcommand{\Gl}{Gl.\ }
\newcommand{\Abb}{Abb.\ }
\newcommand{\Tab}{Tab.\ }


\begin{document}
\LabratoryName{FM.TES}{Tunneleffekt bei Supraleitern}
\ProtocolAuthor{Eric}{Bertok}{eric.bertok@stud.uni-goettingen.de}
\Assistant{Christoph}{Meyer}
\ResearchFocus{Festkörper- und Materialphysik (M.phy.403)}
% In der naechsten Version beruecksichtigt
%\Collaborator{Vorname}{Nachname}{email}
\ConductedOn{25}{2}{2017}
\date{\today}
% eines von beiden
\CopyNotWanted
%\CopyWanted

\pagenumbering{roman}
\maketitle

%\begin{otherlanguage}{english}
%\end{otherlanguage}

\tableofcontents

\clearpage
\pagenumbering{arabic}

\section{Einleitung}
\label{sec:einleitung}
\section{Theorie}
\label{sec:theorie}
\subsection{Grundlagen: Zustandsdichte, Fermiverteilung, Besetzungsdichte}
Die Zustandsdichte $D(E)$ gibt die Anzahl der erlaubten Mikrozustände pro
Energieintervall $\d E$ an. Sie ist für kontinuierliche Energien allgemein
definiert durch $D(E):=\int \frac{\d^d k}{(2
\pi)^d}\delta\left(E-E(\vec{k})\right)$ [cit],
wobei $\vec k$ die Wellenzahl und $d$ die Dimension (für unseren Fall 3) ist.
$E(\vec k)$ ist die Dispersionsrelation der Teilchen.
Für freie Elektronen ergibt sich [cit]:
\begin{equation}
    \label{eq:freielek}
   D_N(E_{\vec{k}})=\frac{\left(2m^*\right)^\frac{3}{2}}{2\pi ^2\hbar ^3}\sqrt{E}, 
\end{equation}
wobei $m^*$ die sog. effektive Masse der Elektronen ist. Diese ist für freie
Elektronen gleich ihrer tatsächlichen Masse, berücksichtigt bei einem
Festkörper aber zusätzlich den Einfluss des Kristallpotentials.

Die Fermi-Dirac Verteilung gibt die mittlere thermische Energiebesetzung von Spin-1/2
Teilchen in Abhängigkeit der Temperatur $T$ und des chemischen Potentials $\mu$
an. Sie lautet $f_T(E)=\frac{1}{\exp{\beta(E-\mu)-1}}$, wobei
$\beta=\frac{1}{k_b T}$ der Boltzmann- Faktor ist. Für $T=0$ ist sie eine
Stufenfunktion [pic]: Unterhalb der sog. Fermienergie $E_F=\mu$ sind alle
Zustände mit Wahrscheinlichkeit 1 besetzt, während alle oberhalb von $E_F$
unbesetzt sind. Dies bezeichnet man als ``Fermikante''.

Die Besetzungsdichte, also die Dichte der tatsächlich besetzten Zustände pro
Energieintervall erhält man als
Produkt aus Zustandsdichte und Fermiverteilung: $n(E)=D(E)f_T(E)$. Hierdurch
erreicht man eine Trennung der systemabhängigen Größen in $D(E)$ und der
systemunabhängigen Thermodynamik in $f_T(E)$.

\subsection{Tunneleffekt}
Der Tunneleffekt ist ein quantenmechanischer Effekt, bei dem ein Teilchen eine
Potentialbarriere endlicher Höhe und breite durchqueren kann, obwohl sie höher
als die Energie des Teilchens ist. Dies kommt dadurch zustande, dass die
Wellenfunktion jenseits der Barriere verschieden von Null ist. 
Durch Lösen der Schrödingergleichung mit einer Kastenbarriere erhält man,
dass die Tunnelwahrscheinlichkeit exponentiell mit der Breite $d$ der Barriere
abfällt: [cit]
\begin{align}
    \label{eq:tunnelamp}
    A_\text{tunnel}\propto e^{-\sqrt{(2m/\hbar^2 (V_0-E)}}e^{-d}.
\end{align}
Jenseits der Barriere das Potential wieder Null. Somit ist die Wellenlänge und damit die Energie des getunnelten
Teilchens unverändert. Lediglich die Wahrscheinlichkeit des Tunnelns nimmt ab.
In unserem Versuch, in dem ein Aluminium- Aluminiumoxid- Blei- Kontakt
verwendet wird, ist die Potentialbreite durch die Dicke der
Aluminiumoxidschicht gegeben. Die Höhe des Potentialwalls ist gegeben durch die
Differenz der Fermienergien der beiden Leiter Al und Pb und der des
Aluminiumoxids.
Weiterhin ist zu beachten, dass ein Teilchen nur dann durch die Barriere
tunneln kann, wenn es auf der anderen Seite freie Zustände vorfindet. Ist dies
nicht der Fall, kommt es zur exponentiellen Abnahme der Amplitude. Auf diese
Weise kann man Schlüsse über die Zustandsdichte der betrachteten Materialien
ziehen, was in diesem Versuch benutzt wird, um die Energielücke von
Supraleitern zu vermessen.

\subsection{mikroskopische Theorie der Supraleitung: Cooper- Paare}
\begin{figure}[]
    \begin{center}
        \includegraphics[scale=0.7]{fig/kreis.png}
    \end{center}
    \caption{eq: Impulsschalen im k-Raum. Impulserhaltung wird erfüllt für den
    Überlapp der beiden Schalen \cite[S.\;309]{enss2011tieftemperaturphysik}.}
    \label{fig:kreis}
\end{figure}
Die BCS Theorie, welche 1957 von Badeen, Cooper und Schrieffer entwickelt
wurde, war die erste erfolgreiche mikroskopische Theorie der Supraleitung
\cite[Kap.\;2]{buckel2013supraleitung}. Die positiven Atomrümpfe werden durch die
negativ geladenen Elektronen polarisiert. Da die Atomrümpfe vergleichsweise
schwer sind, haben sie eine geringe Schwingungsfrequenz. Dadurch kommt es zu
einer retardierten Polarisierung. Ein anderes Elektron spürt diese
Polarisierung also, wenn das erste bereits eine große Distanz zurückgelegt hat.
Aufgrund dessen kann man die Coulombabstoßung zwischen diesen beiden Elektronen
vernachlässigen. Mathematisch lässt sich diese anziehende Wechselwirkung
zwischen zwei Elektronen als ein Austausch eines virtuellen Phonons
beschreiben. Für eine detaillierte Ausarbeitung im Formalismus der zweiten
Quantisierung siehe \zB \cite[Kap.\;10.4]{enss2011tieftemperaturphysik}.
Wenn $\vec q$ der Impuls des Phonons und $\vec k_1$, $\vec k_2$ die der beiden Elektronen
sind, so gilt nach dem Austausch des Phonons
\cite[S.\;308]{enss2011tieftemperaturphysik}
\begin{align}
    \label{eq:phononimp}
    \vec k_1'=\vec k_1 +\vec q, \quad \vec k_2'=\vec k_2 -\vec q,
\end{align}
der Gesamtimpuls $\vec K=\vec k_1 + \vec k_2$ bleibt also erhalten.
Aufgrund der Fermiverteilung sind für die Elektronen nur Energien oberhalb der
Fermikante zugänglich. Approximiert man die Phononenfrequenz mit der
Debyefrequenz $\omega_D$ [cit], so liegen die freien Zustände innerhalb der
Energieschale $[E_F,E_F+\hbar \omega_D]$ oder äquivalent im k-Raum innerhalb
der Schale mit Dicke $\frac{m\omega_D}{\hbar k_F}$, wobei $k_F$ der
Fermiwellenvektor ist. Für beide Elektronen sind diese Schalen in Abb.
\ref{fig:kreis} zu sehen. Impulserhaltung ist erfüllt für alle Vektoren im
Überlapp beider Schalen. Dieser ist maximal für $\vec k_1 = -\vec k_2$. Das
Fermi Prinzip verlangt außerdem, dass beide Elektronen entgegengesetzten Spin
besitzen. Das Paar ${\vec k,\uparrow},{-\vec k,\downarrow}$ bezeichnet man als
Cooper-Paar (CP). Cooper-Paare sind effektive Bosonen. Durch die Attraktion von
Elektronenpaaren wird die Gesamtenergie verringert. Es kommt zu einer
makroskopischen Besetzung, bei der im Grundzustand alle Cooper-Paare im exakt gleichen
Quantenzustand sind \cite[Kap. 2.2]{buckel2013supraleitung}. Die Gesamtheit
aller Cooper-Paare wird deswegen als eine makroskopische Wellenfunktion
beschrieben.
\subsection{Grundzustand und Quasiteilchenanregungen}
Der im vorherigen Abschnitt beschriebene makroskopische Zustand aus
Cooper-Paaren bildet aufgrund der attraktiven Wechselwirkung den neuen
Grundzustand. Möchte man den Supraleiter in einen angeregten Energiezustand
bringen, so muss man zuerst eine Energie $\Delta$ aufbringen, um ein CP
aufzubrechen. Anschließend muss das freie Elektron auf ein höheres
Energieniveau gebracht werden, was wiederum eine Energie von $\Delta$ benötigt
\cite{tidecks1990current}. $\Delta$ Ist die Energielücke des Supraleiters,
welche hier gemessen werden soll. Die Elektronen, die bei der Aufbrechung eines CP's
entstehen, sind keine freien Elektronen, sondern Anregungungen im
Mehrelektronensystem, sogenannte Quasiteilchen. Mit Quasiteilchen können
komplizierte Mehrteilchenphänomene als Einteilchenphänomene beschrieben werden.
Aus der BCS-Theorie erhält man durch Aufstellen der freien Energie
$F=E_\text{kin}+E_\text{pot}-TS$ und deren Minimierung Ausdrücke für die
für die Anregungsenergie $E_{\vec{k}}$ der Quasiteilchenanregungen und für die
Zustandsdichte von Supraleitern $D_S(E_{\vec{k}})$ in Abhängigkeit der Energielücke: \cite[Kap.\;10.4]{enss2011tieftemperaturphysik}:

\begin{align}
    \label{eq:BCS}
    E_{\vec k}&=\sqrt{\eta_{\vec{k}}^2+\Delta^2}\\
    D_S(E_{\vec{k}})&=D_N(E_{\vec{k}})\frac{|E_{\vec k}|}{\sqrt{E_{\vec
    k}^2-\Delta^2}},\quad |E_{\vec k}|\geq\Delta_{T=0}.
\end{align}
$\eta_{\vec k}$ ist die kinetische Energie der Elektronen gemessen von der
Fermienergie. Die Quasiteilchen verhalten sich je nach kinetischer Energie wie
Löcher, Elektronen oder Mischzustände. Für $T=T_c$ ist die kinetische Energie
der Cooper-Paare genug, um diese aufzubrechen und der Supraleiter geht in den
normalleitenden Zustand ohne Energielücke über.

\subsection{Tunnelkennlinien von Normalleitern und Supraleitern}
\begin{figure}[]
    \begin{center}
        \includegraphics[width=\textwidth]{fig/tunnel.png}
    \end{center}
    \caption{Tunneleffekt zwischen zwei Normalleitern (a), zwischen einem
    Normalleiter und einem Supraleiter bei $V=0$ (b) und zwischen einem
    Normalleiter uns Supraleiter bei $U=U_c=\Delta/e$ \cite{dem}
   .}
    \label{fig:tunnel}
\end{figure}
Im Versuch werden sowohl ein Normalleiter-Normalleiter (NL/NL) und ein
Normalleiter-Supraleiterkontakt (NL/SL) verwendet. Die Situation eines NL/NL
Kontaktes ist in Abb. \ref{fig:tunnel} (a) zu sehen. Die Zustandsdichte eines
Normalleiters kann in der nähe der Fermienergie als konstant angenommen werden.
Liegt keine Spannung an, so tunneln im Mittel gleich viele Elektronen von links
nach rechts und umgekehrt. Der Strom ist null. Bei Anlegen einer Spannung
verschiebt sich das chemische Potential und Elektronen tunneln in Richtung
niedrigeren chemischen Potentials in die zugänglichen freien Zustände.
Man erwartet eine Proportionalität zwischen der anliegenden Spannung und dem
gemessenen Tunnelstrom, also Ohm'sches verhalten $U=RI$.
Für einen NL/SL Kontakt liegt die Fermienergie des Normalleiters in der Mitte
der Energielücke des Supraleiters. Somit kann kein Tunnelstrom fließen, da
keine freien Zustände unterhalb $E_F$ vorhanden sind (\ref{fig:tunnel} (b) ). Erst bei einer Spannung
\begin{figure}[]
    \begin{center}
        \includegraphics[scale=0.5]{fig/kennlinie.png}
    \end{center}
    \caption{Tunnelkennlinie eines NL/NL Kontaktes (a), eines NL/SL Kontaktes
    bei $T=0$ (b) und eines NL/SL Kontaktes bei $0<T<T_c$ (c) \cite[S:\;72]{buckel2013supraleitung}.}
    \label{fig:kennlinie}
\end{figure}
von $U=\Delta/e$ können Elektronen in den Supraleiter Tunneln und man misst
einen Tunnelstrom. Dieser steigt sehr steil an, da die Zustandsdichte des
Supraleiters eine Van Hove-Singularität aufweist [cit] (\ref{fig:tunnel} (c) ).
Die Tunnelstrom- Kennlinien sind in Abb. \ref{fig:tunnel} für die verschiedenen
Fälle gezeigt. Für $T\neq 0$ ist die Energielücke kleiner als bei $T=0$.
Außerdem können durch thermische Fluktuationen bereits früher Elektronen in den
Supraleiter tunneln. Dies führt zu einer Aufweichung der Kennlinie
(\ref{fig:kennlinie} (c) ).


\section{Durchführung}
\label{sec:durchfuehrung}
\subsection{Herstellung der Probe}
\begin{figure}[]
    \centering
    \includegraphics[scale=0.5]{fig/aufbau.png}
    \caption{Versuchsaufbau zur Herstellung der Probe. Oben: Substrathalter und
        Schwingungsquarzwaagen.
    Unten: Drehblende [cit].}
    \label{fig:aufbau}
\end{figure}
Zunächst wird die Probe hergestellt. Dazu wird ein SiO$_2$-Wafer verwendet,
auf den die Tunnekontakte draufgedampft werden. Dieser wird zunächst in die
Halterung der Vakuumglocke eingespannt. Jetzt muss die Schablone für das
Aufdampfen montiert werden, wobei auf die korrekte Orientierung zu achten ist.
Desweiteren muss die gesamte Gestell nach oben gedreht werden, da zuerst
gesputtert wird.
Ist alles festgeschraubt, folgt nun die Evakuierung der Glocke, indem zuerst
die Vorpumpe benutzt wird und anschließend mit der
Turbopumpe ein Vakuum von ca. $2\times10^{-5}\;$mbar erzeugt wird.
Zum Sputtern muss die Glocke mit ca. 10$^{-2}$\;bar gefüllt werden.
Nun wird eine ca. 100\;nm dicke Goldschicht auf das Substrat gesputtert. Die
Dicke wird dabei durch eine Schwingungsquarzwaage aufgenommen. Die Anlage
berechnet automatisch durch die sich verändernde Schwingungsfrequenz des
Quarzes die Masse und mit der Dichte die Schichtdicke. Beim Sputtern wird das
Argon ionisiert und die Ionen werden mit einer S~annung von bis zu 1\;kV auf
die Goldfolie beschleunigt. Dadurch werden Goldatome herausgeschlagen, welche
sich auf dem Substrat ablagern. 
\begin{figure}[]
    \centering
    \includegraphics{fig/magneton.png}
    \caption{Schema des Magnetons, welches zum Sputtern verwendet wird [cit].}
    \label{fig:magneton}
\end{figure}
Nun muss das Aluminium aufgedampft werden.  Dafür wird das Ag wieder abgesaugt
und der Probenhalter wird um 180$^\circ$ nach unten gedreht. Die Schablone wird
in die nächste Stellung gedreht und das Aluminium wird durch einen Wolframdraht
erhitzt. Wolfram wird deswegen gewählt, weil er eine besonders hohe
Schmelztemperatur besitzt. Es muss zunächst langsam erhitzt werden, da die
Oxidschicht des Aluminiums zuerst weggedampft werden soll. Während dieses
Vorwärmens muss die Schutztblende über der Probe sein, damit sie kein
Aluminiumoxid abbekommt. Wenn das Aluminium geschmolzen ist, kann die Scheibe
entfernt werden und der Aufdampfvorgang beginnt. Vorher muss die
Schwingungsquarzwaage wissen, dass nun Aluminium verwendet wird.
Die Al-Schichtdicke soll ungefähr bei 50\;nm liegen.
Jetzt kommt der kritischste Punkt des Versuches, nämlich die Herstellung der
Tunnelbarriere. Hierfür wird für eine kurze Zeit eine besonders dosierte Menge
Luft eingelassen, sodass das Aluminium oberflächlich oxidiert. Die genaue
Luftmenge und die Belüftungszeit liegen als Rezept im Labor vor.
Nach der Oxidation wird wieder ein Vakuum erzeugt. Jetzt kann als letzter
Schritt das Blei aufgedampft werden. Dafür wird wieder zunächst die Blende
weitergedreht und anschließend das Blei verdampft. Es soll eine Schichtdicke
von ca. 300\;nm aufgedampft werden.

Als letztes wird die Probe mithilfe eines Mikroskops untersucht. Hierbei ist
insbesondere darauf zu achten, dass die Kontakte unversehrt sind. In unserem
Fall waren Einkerbungen höchstens so breit wie 1/3 der Kontaktbreite und
sollten somit in Ordnung sein.
\begin{figure}[]
    \centering
    \includegraphics{fig/vakuum.png}
    \caption{Vakuumanlage mit Vorpumpe und Turbopumpe [cit].}
    \label{fig:vakuum}
\end{figure}

\subsection{Einbau der Probe und Messung des Tunnelstroms}
\begin{figure}[]
    \centering
    \includegraphics{fig/kryostat.png}
    \caption{He-Verdampfungskryostat [cit].}
    \label{fig:kryostat}
\end{figure}
Das Substart wird zunächst in die Messsonde eingebaut. Dafür werden die sechs
Kontakte mit Indiumplättchen mit der Messsonde verbunden. Indium ist ein
besonders weiches Metall, wodurch ein guter Kontakt mit den Kontakten der Sonde
sichergestellt wird. Jetzt sollten die Kontakte mit einem Amperemeter vermessen
werden. Alle Widerstandswerte unserer Probe waren in diesem Zustand wie zu
erwarten.
Die Messsonde kann jetzt in das Kryostat eingebaut werden. Anschließend wird
die Messung des Tunnelstroms bei Zimmertemperatur durchgeführt. Sowohl Blei als
auch Aluminium sind also Normalleiter. Die Messung geschieht automatisch am
Computer. Hierbei wird eine sogenannte Vierpunktmessung verwendet. Diese
eliminiert den Leitungs- und den Kontaktwiderstand und ist dadurch besonders
geiignet für kleine Tunneströme.
Mit Origin können die gemessenen werte direkt angezeigt werden.
Bereits hier hat sich gezeigt, dass unsere Messwerte für die Spannung
ungefähr um 2 Größenordnungen zu hoch sind.
Nun wird das He-Verdampfungskryostat für die SL/NL-Messung vorbereitet. Dazu wird zunächst mit einer Vakuumpumpe das doppelwandige Gefäß evakuiert. Anschließend wird flüssiger
Stickstoff in die Außenwände gefüllt. Dieses kühlt das Gefäß vor und dient als
zusätzlicher Wärmeisolator für das flüssige Helium. Mit dem flüssigen Helium
wird die Probe auf $4.2$\;K heruntergekühlt. Bei dieser Temperatur ist das Blei
bereits supraleitend. Bei dieser Temperatur wird die erste Tunnelkennlinie
aufgenommen. Auch hier haben unsere Messungen grundsätzlich falsche Ergebnisse
geliefert. Um die Temperaturabhängigkeit der Kennlinien zu untersuchen wird als
letztes der Dampfdruck erniedrigt, wodurch Temperaturen bis zu 1.5\;K erreicht
werden. Hier könnten sich bereits erste Supraleitende eigenschaften des
Aluminiums herauszeichnen. Da der Druck kontinuierlich erniedrigt wird, sind
die hier gemessenen Kennlinien in wahrheit temperaturabhängig. Es wird also ein
Temperatursweep durchgeführt: Kurz vor der gewünschten Temperatur wird die
Messung gestartet. Diese dauert dann 30 Sekunden. Währenddessen sinkt der Druck
weiter. Aus der Dampfdruckkurve von Helium, welche am Kryostat vermerkt ist,
kann dadurch ein Temperaturintervall abgeschätzt werden, welcher als Fehler in
die Auswertung eingeht.

\section{Auswertung}
\label{sec:auswertung}
\subsection{NL/NL-Kennlinie bei Raumtemperatur}
\begin{figure}[]
    \centering
    \includegraphics[width=\textwidth]{fig/1.pdf}
    \caption{Bei Raumtemperatur gemessene Strom-Spannungskurve.}
    \label{fig:1}
\end{figure}
\subsection{SL/NL-Kennlinien}
\begin{figure}[]
    \centering
    \includegraphics[width=\textwidth]{fig/2.pdf}
    \caption{Bei Raumtemperatur gemessene Strom-Spannungskurve.}
    \label{fig:2}
\end{figure}
\begin{figure}[]
    \centering
    \includegraphics[width=\textwidth]{fig/3.pdf}
    \caption{Bei Raumtemperatur gemessene Strom-Spannungskurve.}
    \label{fig:3}
\end{figure}
\begin{figure}[]
    \centering
    \includegraphics[width=\textwidth]{fig/4.pdf}
    \caption{Bei Raumtemperatur gemessene Strom-Spannungskurve.}
    \label{fig:4}
\end{figure}


\section{Diskussion}

\begin{itemize}
    \item 1:NL/NL kleinere Steigung da Messung bei Raumtemperatur (Steigung 1/R)
    \item 2:hohere temp aufgeweichter, asymptote steiler, da kleinere temp.
    \item 3: kleine temp, maximum eindeutig, T=0 wäre unendich hoch
    \item 3: maximum mehr oder weniger immer an gleicher stelle, deswegen grobe
        auswertung.
    \item 3: Keine Fehlerbalken, da der Fehler durch das plateau gegeben
        anstatt fehler der einzelmessungen
    \item 4: korrigierten Werte liegen alle näher an BCS-Theorie, sehr große
        Fehler, kommen durch grobe abschätzung der Maxima. Aber Qualitativ
        richtig. Verbesserten Werte schneiden alle Theoriekurve.
    \item korrigierte werte haben weniger im mittel 17\% abweichung,
        unkorrigierte 31\%.
\end{itemize}

\clearpage

%\appendix
%
%\section{Anhang}
%
%
%\ldots
%
%\clearpage

% nicht unbedingt erforderlich
%\listoffigures
% nicht unbedingt erforderlich
%\listoftables

\bibliography{literatur}

\end{document}

